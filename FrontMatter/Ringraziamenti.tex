% ------------------------------------------------------------------------ %
% !TEX encoding = UTF-8 Unicode
% !TEX TS-program = pdflatex
% !TEX root = ../Tesi.tex
% !TEX spellcheck = it-IT
% ------------------------------------------------------------------------ %
%
% ------------------------------------------------------------------------ %
% 	RINGRAZIAMENTI
% ------------------------------------------------------------------------ %
%
\cleardoublepage
%
\phantomsection
%
\pdfbookmark{Ringraziamenti}{ringraziamenti}
%
\chapter*{Ringraziamenti}
%
Chi mi conosce probabilmente non si aspetta una pagina del genere perchè non sono il tipo di persona che esterna spesso quello che pensa. Ma oggi è un giorno speciale e vorrei cercare di ringraziare tutte le persone che mi hanno accompagnato in questo percorso. Se sono arrivato qui è anche grazie alla vostra presenza.  

% \setlength{\parindent}{5ex} 
In primis naturalmente \emph{Marianna}, sempre al mio fianco. È difficile spiegare a parole quanto il tuo supporto sia stato determinante in questo cammino. In tutti questi anni hai dovuto sentire le mie lamentele sui problemi più disparati e (secondo me l'hai sempre pensato) più assurdi. Mi hai sempre dato il tuo supporto, il tuo affetto e una parola di conforto. Abbiamo cominciato insieme tanti anni fa e siamo ancora qui. Scusa per tutti i problemi che ti ho creato. Semplicemente grazie, \emph{Amore}.

La mia \emph{famiglia}. Senza di voi non potrei essere qui ed il vostro appoggio è stato costante e solido, pronto ad accettare la distanza e l’assenza da casa del vostro ometto. Mi avete sempre permesso di andare dritto per la mia strada senza dover pensare a null’altro che alla mia carriera, ai miei legami, alle mie distrazioni
ed alla mia felicità (anzi quello che si faceva più problemi di tutti ero sempre e solo io).

Una persona che ho incontrato in quest'avventura quasi per caso, ma che è diventata uno dei miei punti di riferimento più grandi. Una vera fortuna conoscerti non saprei come definirla altrimenti. A \emph{Chiara}. Grazie per tutto. E ovviamente non mi posso dimenticare di \emph{Aurora}. Impossibile dividervi. Anche tu mi hai aiutato, anche con 4 semplici chiacchiere vicino al tavolo per svagarmi 5 minuti con il solito \enquote*{problema al pc}, a sentire di meno i peggiori momenti di stress (so che è la prima volta che lo senti dopo tutti questi anni).

Ai miei coinquilini \emph{Fabio}, \emph{Federico}, \emph{Mattia}, \emph{Valerio}. Avete visto il peggio di me e non avete mai avviato una votazione per cacciarmi di casa (il che è tutto dire). Avete condiviso i miei ripassi fino all'ultimo minuto, i miei "e se poi me lo chiede" e, quando tornavo a casa, "alla fine me l'ha chiesto veramente che palle" (lo so che non l'ho mai detto in questo modo ma capitemi).Grazie a tutti voi per aver sempre tentato di farmi ridere con qualche battuta, per le pause caffè e per tutto quello che non sto scrivendo.

A tutti i miei altri coinquilini mancati. \emph{Vincenzo}, \emph{Ilario}, \emph{Giulia}, \emph{Giulia}, \emph{Sara}, \emph{Andrea}, \emph{Dante} e a tutte quante le altre persone che per un motivo o per l'altro sono arrivate nella mia vita. Abbiamo condiviso progetti, nottate, spetteguless, caffè, pucce, panini, ansie da pre-esame, ansie da esame, ansie da post-esame, serate in giro (poche). Grazie per tutto.

Al gruppo di Cyber Security dell’UNIVPM, tra i quali sento il dovere di menzionare in particolare (mi permetto di citarvi per nome) i professori Luca, Marco e Franco. Grazie per il costante aiuto, per i consigli e per gli incoraggiamenti in questa opportunità che mi avete concesso. 

Grazie infine a tutti voi che siete venuti qui ad ascoltarmi o anche solo per un brindisi, a voi che per un motivo o per l'altro non siete potuti essere qui anche se avreste voluto e a tutti voi che per motivi di spazio non ho citato esplicitamente anche se avrei voluto/dovuto. Grazie.
%

\bigskip
%
\noindent\textit{\myLocation, \myTime}
\hfill C.~P.
%
