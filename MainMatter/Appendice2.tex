% ------------------------------------------------------------------------ %
% !TEX encoding = UTF-8 Unicode
% !TEX TS-program = pdflatex
% !TEX root = ../Tesi.tex
% !TEX spellcheck = it-IT
% ------------------------------------------------------------------------ %
%
% ------------------------------------------------------------------------ %
% 	NOME APPENDICE 2
% ------------------------------------------------------------------------ %
%
\chapter{Schema transazioni Quorum}
%
\label{cap:transaction}
%
La figura mostra come viene processata una transazione privata in Quorum:
%
\begin{figure}[H]
	%
	\centering
	%
	\includegraphics[width=.9\textwidth]{Quorum/transaction}
	%
	\caption{Schema di una transazione privata Quorum}
	%
	\label{fig:schema di una transazione privata quorum}
	%
\end{figure}
%
\begin{enumerate}
	\item il party A invia una transazione al proprio nodo Quorum, specificando il payload della transazione e settando il \enquote*{privateFor} con la chiave pubblica per i gruppi A e B.
	\item il nodo Quorum del party A passa la transazione al relativo gestore delle transazioni, richiedendogli di memorizzare il payload.
	\item il transaction manager del party A chiama il proprio Enclave associato per validare il mittente e crittare il payload.
	\item l'Enclave del party A controlla la chiave privata\footnote{\gls{asimmetrica}} del party A e, una volta validata, effettua la conversione della transazione. Questo comporta: 
	      \begin{enumerate}
	      	\item generazione di chiave simmetrica e random nonce.
	      	\item crittografia del payload della transazione e del nonce con la chiave simmetrica dello step precedente.
	      	\item calcolo SHA3-512 del payload del passo precedente.
	      	\item iterazione attraverso la lista dei destinatari della transazione, in questo caso i party A e B, crittando la chiave simmetrica dello step precedente con la chiave pubblica del destinatario (PGP). 
	      	\item viene ritornato il payload crittografato allo step 4.2, l'hash dello step 4.3 e la chiave crittografata (per ogni destinatario) dello step precedente al transaction manager.
	      \end{enumerate}
	\item il transaction manager del party A immagazzina il payload (criptato con la chiave simmetrica) e la chiave simmetrica criptata utilizzando l'hash come indice e poi trasferendo in modo sicuro: hash, payload criptato e la chiave simmetrica criptata con la chiave pubblica del party B al transaction manager del party B. Quest'ultimo risponderà con una risposta di tipo ack/nack. Se il party A non riceve risposta o riceve nack dal party B allora la transazione non sarà propagata nella rete.
	\item una volta che la trasmissione dei dati al transaction manager del party B ha avuto successo il transaction manager del party A ritorna l'hash al nodo Quorum, il quale hash va a rimpiazzare il payload originale della transazione e cambia il valore \emph{V} della transazione in 37 o 38. Questo valore indicherà agli altri nodi che questo hash rappresenta una transazione privata con un payload associato criptato.
	\item a questo punto la transazione è propagata al resto della rete utilizzando il protocollo P2P standard di Ethereum.
	\item un blocco contenente la transazione AB è creato e distribuito ad ogni party sulla rete.
	\item Elaborando il blocco tutti i party proveranno a processare la transazione. Ogni nodo Quorum riconoscerà un valore \emph{V} di 37 o 38, identificandola quindi come transazione il cui payload deve essere decrittato, ed effettuerà una chiamata al proprio transaction manager per determinare se possiedono la transazione (utilizzando l'hash come indice per cercare).
	\item dato che il party C non possiede la transazione riceverà un messaggio \enquote*{NotRecipient} e salterà la transazione non aggiornando il suo stateDb privato. Invece la ricerca da parte dei party A e B avrà esito positivo identificando quindi il fatto che posseggono la transazione e procederanno a chiamare il proprio Enclave passando il payload criptato, la chiave simmetrica criptata e la firma.
	\item Enclave valida la firma e poi decritta la chiave simmetrica utilizzando la chiave privata del party (posseduta da Enclave) , decritta il payload della transazione utilizzando la chiave simmetrica appena ottenuta e ritorna il payload decrittato al transaction manager.
	\item il transaction manager dei party A e B allora invia il payload decrittato all'EVM per l'esecuzione del codice del contratto. Questa esecuzione aggiornerà solo il private stateDB del nodo Quorum. Una volta che il codice è stato eseguito, viene eliminato per far sì che non sia disponibile a meno di rieseguire il processo di decrittazione
\end{enumerate}