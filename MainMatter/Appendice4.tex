% ------------------------------------------------------------------------ %
% !TEX encoding = UTF-8 Unicode
% !TEX TS-program = pdflatex
% !TEX root = ../Tesi.tex
% !TEX spellcheck = it-IT
% ------------------------------------------------------------------------ %
%
% ------------------------------------------------------------------------ %
%   NOME APPENDICE 4
% ------------------------------------------------------------------------ %
%
\chapter{Script di configurazione}
%
\label{cap:codici}
%
Questo capitolo conterrà tutti gli script di configurazione, utilizzati nell'ambito dell'implementazione dell'applicazione e citati nel capitolo \autoref{cap:implementazione}.
%
\section{Vagrant}
%
\label{cap:vagrantConfig}
%
Il VagrantFile utilizzato nello sviluppo:\newline
\begin{center}
	\begin{lstlisting}[language=ruby,caption={VagrantFile del progetto},captionpos=b,frame=lines,basicstyle=\linespread{0.9}\small]
Vagrant.configure(2) do |config|
  config.vm.box = "ubuntu/xenial64"
  config.vm.provision :shell, path: "vagrant/bootstrap.sh"
  config.vm.network "forwarded_port", guest: 22000, host: 22000
  config.vm.network "forwarded_port", guest: 22001, host: 22001
  config.vm.network "forwarded_port", guest: 22002, host: 22002
  config.vm.network "forwarded_port", guest: 22003, host: 22003
  config.vm.network "forwarded_port", guest: 22004, host: 22004
  config.vm.network "forwarded_port", guest: 22005, host: 22005
  config.vm.network "forwarded_port", guest: 22006, host: 22006
  config.vm.provider "virtualbox" do |v|
    v.memory = 4096
  v.gui = false
  end
end
	\end{lstlisting}
\end{center}
%
Invece il file bootstrap.sh utilizzato per configurare le dipendenze software della blokchain è il seguente:
\newline
\begin{center}\begin{lstlisting}[language=sh,caption={Script di bootstrap utilizzato dal Vagrantfile},captionpos=b,frame=lines,basicstyle=\linespread{0.9}\small]
	#!/bin/bash
							      
	set -eu -o pipefail
							      
	# install build deps
	add-apt-repository ppa:ethereum/ethereum
	apt-get update
	apt-get install -y build-essential unzip libdb-dev libsodium-dev zlib1g-dev libtinfo-dev solc sysvbanner wrk
							      
	# install constellation
	wget -q https://github.com/jpmorganchase/constellation/releases/download/v0.0.1-alpha/ubuntu1604.zip
	unzip ubuntu1604.zip
	cp ubuntu1604/constellation-node /usr/local/bin && chmod 0755 /usr/local/bin/constellation-node
	cp ubuntu1604/constellation-enclave-keygen /usr/local/bin && chmod 0755 /usr/local/bin/constellation-enclave-keygen
	rm -rf ubuntu1604.zip ubuntu1604
							      
	# install golang
	GOREL=go1.7.3.linux-amd64.tar.gz
	wget -q https://storage.googleapis.com/golang/$GOREL
	tar xfz $GOREL
	mv go /usr/local/go
	rm -f $GOREL
	PATH=$PATH:/usr/local/go/bin
	echo 'PATH=$PATH:/usr/local/go/bin' >> /home/ubuntu/.bashrc
							      
	# make/install quorum
	git clone https://github.com/jpmorganchase/quorum.git
	pushd quorum >/dev/null
	git checkout tags/v1.1.0
	make all
	cp build/bin/geth /usr/local/bin
	cp build/bin/bootnode /usr/local/bin
	popd >/dev/null
							      
	# copy examples
	cp -r /vagrant/examples /home/ubuntu/quorum-examples
	chown -R ubuntu:ubuntu /home/ubuntu/quorum /home/ubuntu/quorum-examples
	\end{lstlisting}
	%
\end{center}
%
\section{Blockchain Quorum}
%
\label{cap:quorumCode}
%
Il file \emph{init.sh} è stato riorganizzato in maniera tale da avere all'avvio della blockchain due account sul nodo target 1:
\begin{center}
	\begin{lstlisting}[language=sh,caption={Script di inizializzazione della blockchain Quorum},captionpos=b,frame=lines,basicstyle=\linespread{0.8}\small]
	#!/bin/bash
	set -u
	set -e

	echo "[*] Cleaning up temporary data directories"
	rm -rf qdata
	mkdir -p qdata/logs

	echo "[*] Configuring node 1"
	mkdir -p qdata/dd1/keystore
	cp keys/key1 qdata/dd1/keystore
	cp keys/key2 qdata/dd1/keystore
	geth --datadir qdata/dd1 init genesis.json

	echo "[*] Configuring node 2 as block maker and voter"
	mkdir -p qdata/dd2/keystore
	cp keys/key2 qdata/dd2/keystore
	cp keys/key3 qdata/dd2/keystore
	geth --datadir qdata/dd2 init genesis.json

	echo "[*] Configuring node 3"
	mkdir -p qdata/dd3/keystore
	geth --datadir qdata/dd3 init genesis.json

	echo "[*] Configuring node 4 as voter"
	mkdir -p qdata/dd4/keystore
	cp keys/key4 qdata/dd4/keystore
	geth --datadir qdata/dd4 init genesis.json

	echo "[*] Configuring node 5 as voter"
	mkdir -p qdata/dd5/keystore
	geth --datadir qdata/dd5 init genesis.json

	echo "[*] Configuring node 6"
	mkdir -p qdata/dd6/keystore
	geth --datadir qdata/dd6 init genesis.json

	echo "[*] Configuring node 7"
	mkdir -p qdata/dd7/keystore
	geth --datadir qdata/dd7 init genesis.json
	\end{lstlisting}
\end{center}
Invece, il file start.sh è stato riscritto per avere i nodi configurati secondo le necessità dell'applicazione:
\begin{center}
	\begin{lstlisting}[language=sh,caption={Script di avvio della blockchain Quorum},captionpos=b,frame=lines,basicstyle=\linespread{0.8}\small]
	#!/bin/bash
	set -u
	set -e
	NETID=87234
	BOOTNODE_KEYHEX=77bd02ffa26e3fb8f324bda24ae588066f1873d95680104de5bc2db9e7b2e510
	BOOTNODE_ENODE=enode://61077a284f5ba7607ab04f33cfde2750d659ad9af962516e159cf6ce708646066cd927a900944ce393b98b95c914e4d6c54b099f568342647a1cd4a262cc0423@[127.0.0.1]:33445

	GLOBAL_ARGS="--bootnodes $BOOTNODE_ENODE --networkid $NETID --rpc --rpcaddr 0.0.0.0 --rpcapi admin,db,eth,debug,miner,net,shh,txpool,personal,web3,quorum"

	echo "[*] Starting Constellation nodes"
	nohup constellation-node tm1.conf 2>> qdata/logs/constellation1.log &
	sleep 1
	nohup constellation-node tm2.conf 2>> qdata/logs/constellation2.log &
	nohup constellation-node tm3.conf 2>> qdata/logs/constellation3.log &
	nohup constellation-node tm4.conf 2>> qdata/logs/constellation4.log &
	nohup constellation-node tm5.conf 2>> qdata/logs/constellation5.log &
	nohup constellation-node tm6.conf 2>> qdata/logs/constellation6.log &
	nohup constellation-node tm7.conf 2>> qdata/logs/constellation7.log &

	echo "Bootnode"
	nohup bootnode --nodekeyhex "$BOOTNODE_KEYHEX" --addr="127.0.0.1:33445" 2>>qdata/logs/bootnode.log &
	echo "wait for bootnode to start..."
	sleep 6

	echo "Node 1"
	PRIVATE_CONFIG=tm1.conf nohup geth --datadir qdata/dd1 $GLOBAL_ARGS --rpcport 22000 --port 21000 --unlock "0xed9d02e382b34818e88b88a309c7fe71e65f419d" --password passwords.txt --voteaccount "0x0638e15747$

	echo "Node 2"
	PRIVATE_CONFIG=tm2.conf nohup geth --datadir qdata/dd2 $GLOBAL_ARGS --rpcport 22001 --port 21001 --voteaccount "0x0fbdc686b912d7722dc86510934589e0aaf3b55a" --votepassword "" --blockmakeraccount "0xca8435$

	echo "Node 3"
	PRIVATE_CONFIG=tm3.conf nohup geth --datadir qdata/dd3 $GLOBAL_ARGS --rpcport 22002 --port 21002 2>>qdata/logs/3.log &

	echo "Node 4"
	PRIVATE_CONFIG=tm4.conf nohup geth --datadir qdata/dd4 $GLOBAL_ARGS --rpcport 22003 --port 21003 --voteaccount "0x9186eb3d20cbd1f5f992a950d808c4495153abd5" --votepassword "" 2>>qdata/logs/4.log &

	echo "Node 5"
	PRIVATE_CONFIG=tm5.conf nohup geth --datadir qdata/dd5 $GLOBAL_ARGS --rpcport 22004 --port 21004 2>>qdata/logs/5.log &

	echo "Node 6"
	PRIVATE_CONFIG=tm6.conf nohup geth --datadir qdata/dd6 $GLOBAL_ARGS --rpcport 22005 --port 21005 2>>qdata/logs/6.log &

	echo "Node 7"
	PRIVATE_CONFIG=tm7.conf nohup geth --datadir qdata/dd7 $GLOBAL_ARGS --rpcport 22006 --port 21006 2>>qdata/logs/7.log &

	echo "Nodes Up. Blockchain up. Run 'geth attach qdata/dd1/geth.ipc' to attach to the first Geth node"
	\end{lstlisting}
\end{center}
Nella blockchain appena deployata si può notare come ogni nodo possieda un'istanza di geth (ovvero è un full node)