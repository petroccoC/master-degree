%
% !TEX encoding = UTF-8 Unicode
% !TEX TS-program = pdflatex
% !TEX root = Tesi.tex
% !TEX spellcheck = it-IT
%
% ------------------------------------------------------------------------ %
% Other personal commands
% ------------------------------------------------------------------------ %
%
\newcommand{\myName}{Claudio Petrocco}			% autore
\newcommand{\myMatricola}{1065591}			% matricola
\newcommand{\myTitle}{Gestione delle prescrizioni mediche dematerializzate utilizzando una Blockchain privata}	% titolo
\newcommand{\myUni}{Università Politecnica delle Marche}		% università
\newcommand{\myDegree}{Ingegneria Informatica e dell'Automazione}		% laurea
\newcommand{\myThesis}{Tesi di Laurea Magistrale}	% tipo di tesi
\newcommand{\myDepartment}{Dipartimento dell'Informazione}	% dipartimento
\newcommand{\myProf}{Prof.~Luca~Spalazzi}		% relatore
\newcommand{\myOtherProf}{Ing.~Marco~Baldi}		% eventuale correlatore
\newcommand{\myLocation}{Ancona}			% dove
\newcommand{\myTime}{Dicembre 2017}			% quando
\newcommand{\myAcademicYear}{2016--2017}		% anno accademico
%
% ------------------------------------------------------------------------ %
% Amsmath, amssymb, amsthm
% ------------------------------------------------------------------------ %
%
% Uncomment if necessary
%
% \newenvironment{sistema}%
% {\left\lbrace\begin{array}{@{}l@{}}}%
% {\end{array}\right.}
% %
% % epsilon theta rho phi
% \renewcommand{\epsilon}{\varepsilon}
% \renewcommand{\theta}{\vartheta}
% %\renewcommand{\rho}{\varrho}
% \renewcommand{\phi}{\varphi}
% %
% \renewcommand{\vec}{\mathbf} 
%
% ------------------------------------------------------------------------ %
% Xcolor
% ------------------------------------------------------------------------ %
%
% webcolors
\definecolor{webgreen}{rgb}{0,.5,0}
\definecolor{webbrown}{rgb}{.6,0,0}
%
% Blue
\definecolor{darkbluePoliMi}{rgb}{0,0.18,0.40}	%rgb(0, 46, 103)
\definecolor{midbluePoliMi}{rgb}{0.33,0.47,0.62}	%rgb(84, 121, 157)
\definecolor{lightbluePoliMi}{rgb}{0.53,0.64,0.73}	%rgb(134, 163, 186)
\definecolor{orangePoliMi}{rgb}{1,0.59,0}		%rgb(255, 151, 0)
%
%
% ------------------------------------------------------------------------ %
% Listings
% ------------------------------------------------------------------------ %
%
\lstset{
	basicstyle=\smaller[0]\ttfamily,		% Black & White:
	keywordstyle=\color{RoyalBlue},	% keywordstyle=\color{black}\bfseries,
	commentstyle=\color{webgreen},	% commentstyle=\color{gray},
	stringstyle=\color{webbrown},		% stringstyle=\color{black},
	numbers=left,
	numberstyle=\smaller[2],
	stepnumber=1,
	numbersep=8pt,
	showspaces=false,
	showstringspaces=false,
	showtabs=false,
	breaklines=true,
	frameround=ffff,
	frame=single,
	tabsize=2,
	captionpos=t,
	breakatwhitespace=false,
}
%
% Solution to the encoding issue
\lstset{literate=
	{á}{{\'a}}1 {é}{{\'e}}1 {í}{{\'i}}1 {ó}{{\'o}}1 {ú}{{\'u}}1
	{Á}{{\'A}}1 {É}{{\'E}}1 {Í}{{\'I}}1 {Ó}{{\'O}}1 {Ú}{{\'U}}1
	{à}{{\`a}}1 {è}{{\`e}}1 {ì}{{\`i}}1 {ò}{{\`o}}1 {ù}{{\`u}}1
	{À}{{\`A}}1 {È}{{\'E}}1 {Ì}{{\`I}}1 {Ò}{{\`O}}1 {Ù}{{\`U}}1
	{ä}{{\"a}}1 {ë}{{\"e}}1 {ï}{{\"i}}1 {ö}{{\"o}}1 {ü}{{\"u}}1
	{Ä}{{\"A}}1 {Ë}{{\"E}}1 {Ï}{{\"I}}1 {Ö}{{\"O}}1 {Ü}{{\"U}}1
	{â}{{\^a}}1 {ê}{{\^e}}1 {î}{{\^i}}1 {ô}{{\^o}}1 {û}{{\^u}}1
	{Â}{{\^A}}1 {Ê}{{\^E}}1 {Î}{{\^I}}1 {Ô}{{\^O}}1 {Û}{{\^U}}1
	{œ}{{\oe}}1 {Œ}{{\OE}}1 {æ}{{\ae}}1 {Æ}{{\AE}}1 {ß}{{\ss}}1
	{ç}{{\c c}}1 {Ç}{{\c C}}1 {ø}{{\o}}1 {å}{{\r a}}1 {Å}{{\r A}}1
	{€}{{\EUR}}1 {£}{{\pounds}}1
}
%
% Definizione ambienti per i vari linguaggi
%
\lstnewenvironment{Matlab}{\lstset{language=Matlab}}{}
%
\lstnewenvironment{C++}{\lstset{language=C++}}{}
%
\lstnewenvironment{bash}{\lstset{language=bash}}{}
%
%
% Code listings
%
\addto\captionsitalian{\renewcommand{\lstlistingname}{Codice}}
%
\addto\captionsitalian{\renewcommand{\lstlistlistingname}{Elenco dei codici}}
%
% ------------------------------------------------------------------------ %
% Hyperref
% ------------------------------------------------------------------------ %
%
\hypersetup{
	colorlinks=true,
	linktocpage=true,
	pdfstartpage=1,
	pdfstartview=FitV,
	breaklinks=true,
	pageanchor=true,
	pdfpagemode=UseOutlines,
	bookmarksopenlevel=1,
	hypertexnames=true,
	pdfhighlight=/O,
	urlcolor=webbrown,		
	linkcolor=RoyalBlue,
	citecolor=webgreen,
	pdftitle={\myTitle},		
	pdfauthor={\textcopyright\ \myName, \myUni},
	pdfsubject={},
	pdfcreator={pdfLaTeX},
	pdfproducer={LaTeX with hyperref}
}
%
% ------------------------------------------------------------------------ %
% Graphicx
% ------------------------------------------------------------------------ %
%
\graphicspath{
	{Immagini/}
	{Immagini/Introduzione/}
	{Immagini/Sanitadigitale/}
	{Immagini/Blockchain/}
	{Immagini/Ethereum/}
	{Immagini/Quorum/}
	{Immagini/StrumentiSviluppo/}
	{Immagini/Implementazione/}
}
%
% ------------------------------------------------------------------------ %
% Fancyhdr
% ------------------------------------------------------------------------ %
%
\pagestyle{fancy}			% instead of \pagestyle{header} standard
%
\makeatletter 
%
\renewcommand{\chaptermark}[1]{	% chapter def
	\markboth{\@chapapp\ \thechapter.\ #1}{}} % Chapter / appendix
\makeatother
%
\renewcommand{\sectionmark}[1]{	% section ndef
	\markright{\thesection.\ #1}}
%
\fancyhf{}				% Empty header and footer
%
\fancyhead[LE,RO]{\bfseries}	% numero pagine in alto
%
\fancyhead[LO]{\bfseries\rightmark}	% info section in odd page
%
\fancyhead[RE]{\bfseries\leftmark}	% info chapter in even page
%
\cfoot{\thepage}
%
\renewcommand{\headrulewidth}{0.4pt}	% 
%
\renewcommand{\footrulewidth}{0pt}	% (0pt=hidden)
%
\fancypagestyle{plain}{				% begin chapter style
	\fancyhead{}			% empty header
	\fancyfoot[C]{\bfseries\thepage}		% bold numbers centered
	\renewcommand{\headrulewidth}{0pt}	% no line
}
%
% ------------------------------------------------------------------------ %
% Other
% ------------------------------------------------------------------------ %
%
% Gradiente
\newcommand{\gradiente}[1]{$\nabla #1$}
%
% puntini di omissione [...]
\newcommand{\omissis}{[\dots\negthinspace]}
%
% Eccezioni all'algoritmo di sillabazione
\hyphenation{OpenFOAM}
\hyphenation{Matlab}
\hyphenation{bash}
%
% ------------------------------------------------------------------------ %
% Finezze tipografiche per il Politecnico di Milano
% ------------------------------------------------------------------------ %
%
% Le seguenti modifiche possono essere commentate
% o adeguate ad un'altra università (es. 'Yale Blue'
% per l'università di Yale, 'Rosso Sapienza' per La Sapienza..)
%
% Filetti tabelle colorati
% \arrayrulecolor{darkbluePoliMi}
%
%
% Righe delle note a piè di pagina colorate
\renewcommand{\footnoterule}{%
	\kern -3pt
	{\color{darkbluePoliMi} \hrule width 0.4\textwidth}
	\kern 2.6pt
}
%
% ------------------------------------------------------------------------ %
% Comandi aggiuntivi inseriti 
% ------------------------------------------------------------------------ %
\renewcommand*\thesection{\arabic{section}}		% i numeri delle sezioni passano da #chapter.#section a solo #section
%
\colorlet{punct}{red!60!black}
\definecolor{background}{HTML}{EEEEEE}
\definecolor{delim}{RGB}{20,105,176}
\colorlet{numb}{magenta!60!black}
%
%define json
\lstdefinelanguage{json}{
	basicstyle=\normalfont\ttfamily\linespread{0.9}\small,
	numbers=left,
	numberstyle=\scriptsize,
	stepnumber=1,
	numbersep=7pt,
	showstringspaces=false,
	breaklines=true,
	frame=lines,
	%backgroundcolor=\color{background},
	literate=
	*{0}{{{\color{numb}0}}}{1}
	{1}{{{\color{numb}1}}}{1}
	{2}{{{\color{numb}2}}}{1}
	{3}{{{\color{numb}3}}}{1}
	{4}{{{\color{numb}4}}}{1}
	{5}{{{\color{numb}5}}}{1}
	{6}{{{\color{numb}6}}}{1}
	{7}{{{\color{numb}7}}}{1}
	{8}{{{\color{numb}8}}}{1}
	{9}{{{\color{numb}9}}}{1}
	{:}{{{\color{punct}{:}}}}{1}
	{,}{{{\color{punct}{,}}}}{1}
	{\{}{{{\color{delim}{\{}}}}{1}
	{\}}{{{\color{delim}{\}}}}}{1}
	{[}{{{\color{delim}{[}}}}{1}
	{]}{{{\color{delim}{]}}}}{1},
}%
%define javascript
\definecolor{darkgray}{rgb}{.4,.4,.4}
\definecolor{purple}{rgb}{0.65, 0.12, 0.82}
\lstdefinelanguage{JavaScript}{
	keywords={typeof, new, true, false, catch, function, return, null, catch, switch, var, if, in, while, do, else, case, break},
	keywordstyle=\color{blue}\bfseries,
	ndkeywords={class, export, boolean, throw, implements, import, this},
	ndkeywordstyle=\color{darkgray}\bfseries,
	identifierstyle=\color{black},
	frame=lines,
	basicstyle=\linespread{0.9}\small,
	sensitive=false,
	comment=[l]{//},
	morecomment=[s]{/*}{*/},
	commentstyle=\color{purple}\ttfamily,
	stringstyle=\color{red}\ttfamily,
	morestring=[b]',
	morestring=[b]"
}%
%%%%
\newenvironment{para}[1]
{%here comes what is processed before
	\noindent\textbf{#1}\hspace{1em}\ignorespaces}
{%and here what is processed after
	\par}
%%%%
% COMMAND: \pic
% SYNTAX:  \pic{PATH}!LABEL![SIZE]
% EFFECT:  Draws PATH image scaled to SIZE*columnwidth, eventually with LABEL.
% \DeclareDocumentCommand{\pic}{ m d!! O{0.5} }
% {
% 	\begin{center}\includegraphics[width=#3\columnwidth]{#1}
% 		\IfValueT{#2}
% 		{
% 			\label{#2}
% 		}
% 	\end{center}
% }
